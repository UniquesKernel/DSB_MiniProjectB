\section{Opgave3.15}

Given a signal amplitude vector x = [3 -1 0.6 -0.8  -2 0 0.9];

We first calculate the DFT values by matlab function fft(x).

The results if fft(x) can been seen in figure \ref{fig:opg315} blue dots. Note that only the real values (Amplitudes) are shown for more information see the Matlab file opgave_315.m.

Now we calculate the twiddle-factor for an N=8 point DFT.
$W_8 = e^(-j*(2*pi*m*n/8)$ where $n$ and $m$ are the matrix notations corresponding to the column, row notation.

We now calculate the needed 8 twiddle-factors corresponding to the 8 different position on the unit circle. 
$(W_8)^0 = 1;$\n
$(W_8)^1 = exp((-j*(2*pi/8)));$\n
$(W_8)^2 = exp((-j*(2*2*pi/8)));$\n
$(W_8)^3 = exp((-j*(2*3*pi/8)));$\n
$(W_8)^4 = exp((-j*(2*4*pi/8)));$\n
$(W_8)^5 = exp((-j*(2*5*pi/8)));$\n
$(W_8)^6 = exp((-j*(2*6*pi/8)));$\n
$(W_8)^7 = exp((-j*(2*7*pi/8)));$\n
 
Remember that $(W_8)^n = (W_8)^(n-8)$ and so on. 
The full matrix $W$ can bee seen in the matlab file.

We now calculate the product $x*W$ and get the DFT values also seen in the figure \ref{fig:opg315} red dots. 

We observe that the two different methods of calculating DFT values are not the same, this is expected as the vector is a size of 7 elements and the DFT calculated manually is of N=8.

\begin{figure}[H]
    \centering
    \includegraphics[width=1\textwidth]{matlabStuff/opg315.PNG}
     \caption{Comparison of the matlab DFT (blue) and manual DFT values (red) real values}
    \label{fig:opg315}
\end{figure}
