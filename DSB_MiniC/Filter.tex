\section{Filtre}
This project utilizes two types of filter a FIR (finite inpulse responds) 
filter and an IIR (infinite impulse responds) filter.
The FIR filter is designed and implemented in matlab using the window method.

\subsection{FIR Filter Design}
First step in designing a FIR filter is to design an ideal IIR filter before
trucating it with by multiplying the IIR filter with a finite length window
function. 

By using our spectral analysis from the earlier sections, we qualitatively
decided to make the cutoff frequence $f_{c} = 10$. The sample frequency $f_{s}$
is given from our dataset, $f_{s} = 47.7774$.

Lastly, the filters made order $M = 250$ thus using $250$ filter coefficients. 

\subsubsection{Resolution}
Next step in designing our filter, we determine the frequency resolution
which provides specifications for the FIR transfer function.

\begin{align}
  \label{eq:freqResolution}
  f_{res} &= \frac{f_{s}}{M}
\end{align}

\subsubsection{Transfer function}
Using $f_{res}$ and $f_{c}$, we can determine which frequency bin
corresponds to frequencies below $f_c$.
This must be in done in integer values i.e.\ rounded to 
closes integer value.

\begin{align}
  \label{eq:freqBin}
  f_{bin} = \left\lfloor\frac{f_{c}}{f_{res}}\right\rfloor = 52
\end{align}

In this case the bin number corresponds to $f_c$ is 52, which
we design our lowpass filter around see \autoref{fig:specificatiion}.

These specification help determine which frequencies should be passed and which should be removed. In this case we remove everything above frequency 10.

\begin{figure}[h]
  \centering
  \includegraphics[scale = 0.5]{matlabStuff/Specification_of_transfer_function.png}
  \caption{Specification for our transfor function}%
  \label{fig:specificatiion}
\end{figure}

\newpage

In matlab we can now make our transfer function $h$ using the follwing code. Which when combined with, in our case a hanning window function serves to become our filter which we can run our
data set through.

\begin{figure}[h]
  \centering
  \lstinputlisting[language={matlab}, linerange={27-28,40-40,49-50,58-60}]{matlabStuff/filter.m}
  \caption{matlab code for making a transfer function}%
  \label{}
\end{figure}



Now that we have our transfer function, we can apply it to our data to see if we can reduce the potential noice in data.
